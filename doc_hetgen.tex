\documentclass{article}
\usepackage{arxiv}
\usepackage[utf8]{inputenc} % allow utf-8 input
\usepackage[T1]{fontenc}    % use 8-bit T1 fonts
\usepackage{hyperref}       % hyperlinks
\usepackage{url}            % simple URL typesetting
\usepackage{booktabs}       % professional-quality tables
\usepackage{amsfonts}       % blackboard math symbols
\usepackage{nicefrac}       % compact symbols for 1/2, etc.
\usepackage{microtype}      % microtypography
\usepackage{amsmath}
\usepackage{graphicx}
\usepackage{array}
\usepackage{tabularx}

\title{Isolation of infected people and their contacts is likely to be effective
  against many short-term epidemics}

\author{
  Nathan Geffen \\
  Centre for Social Science Research\\
  University of Cape Town\\
  \texttt{nathan.geffen@alumni.uct.ac.za} \\
  \and
  Marcus Low\\
  Department of Computer Science\\
  University of Cape Town\\
  \texttt{LWXMAR013@MyUCT.ac.za}}

\date{\today}

\usepackage{listings}
\usepackage{color}

% For nicely formatting code
\lstset{frame=tb,
  language=Python,
  aboveskip=3mm,
  belowskip=3mm,
  showstringspaces=false,
  columns=flexible,
  basicstyle={\small\ttfamily},
  numbers=none,
  breaklines=true,
  breakatwhitespace=true,
  tabsize=3
}

\begin{document}

\maketitle

\begin{abstract}

  Background: Isolation of infected people and their contacts may be an
  effective way to control outbreaks of infectious disease, such as influenza
  and SARS-CoV-2. Models can provide insights into the efficacy of contact
  tracing, coupled with isolating or quarantining at risk people.\\
  Methods: We developed an agent-based model and simulated $15,000$ short term
  illnesses, with varying characteristics. For
  each illness we ran ten simulations on the following scenarios: (1) No
  tracing or isolation (None), (2) isolation of agents who have tested positive
  (Isolation), (3) scenario 2 coupled with minimal contact tracing and quarantine
  of contacts (Minimum), (4) scenario 3 with more effective contact tracing
  (Moderate), and (5) perfect isolation of agents who test positive and perfect
  tracing and quarantine of all their contacts (Maximum).\\
  Results: The median total infections of the Isolation, Minimum, Moderate and
  Maximum scenarios were 80\%, 40\%, 17\% and 4\% of the no intervention
  scenario respectively.\\
  Conclusions: Isolation of infected patients and quarantine of their contacts,
  even if moderately well implemented, is likely to substantially reduce the
  number of infections in an outbreak. Randomized controlled trials to confirm
  these results in the real world and to analyse the cost effectiveness of
  contact tracing and isolation during coronavirus and influenza outbreaks are
  warranted.

\end{abstract}

\keywords{SARS-CoV-2 \and Covid-19 \and Agent-based model \and Contact tracing}

\section{Introduction}

Several non-pharmaceutical interventions have been introduced in different
countries to reduce the spread of the SARS-CoV-2 pandemic. Some of these have
included lockdowns, physical distancing rules, prolific use of hand sanitisers,
and mask-wearing \cite{Flaxman2020,Lemaitre2020,Cowling2020}.

In some countries, such as South Korea, a key intervention has been tracing the
contacts of infected people so that they can isolate or quarantine themselves
\cite{Jung2020,Lee2020}. Indeed, preparation for tracking coronavirus outbreaks
was in place in South Korea long before the emergence of Covid-19
\cite{Park2016}.

Isolation refers to the movement of people with the infection being restricted,
voluntarily or legally, while quarantine refers to the movement of people at
risk of infection being restricted, voluntarily or legally. The work described
here does not differentiate between these distinctions. We therefore use
\emph{isolation} to refer interchangeably to isolation or quarantine, whether
voluntarily undertaken or legally enforced.

Besides coronaviruses, contact tracing has also been implemented to control
tuberculosis, hepatitis and HIV and other sexually transmitted infections
\cite{Dennis2018,MacPherson2019,Katzman2019,Stokes1999}. But these are rather
different illnesses to Covid-19, generally infecting patients for longer and
spreading slower through populations. It is also not usually ethical or
practical to expect people with a long-duration infection to isolate.

Contact tracing has to a limited extent also been used to control influenza
outbreaks \cite{Swaan2011,Eames2010}. Contact tracing was a ``critical
intervention'' in the Liberian Ebola epidemic of 2014-2015 and ``represented one
of the largest contact tracing efforts during an epidemic in history'' \cite{Swanson2018}.

This raises the question: How effective is contact tracing coupled with
isolation (CTI) at controlling outbreaks of short duration illnesses such as
those associated with coronaviruses and influenza? And how well
implemented must contact tracing and isolation be? After all, CTI is potentially
a far less costly and intrusive way of controlling dangerous illnesses than
lockdowns, though both together may be necessary in some situations.

It's impossible to provide a precise answer to this question. Infection
outbreaks have practically infinite variations and are extraordinarily complex
and stochastic. As the SARS-CoV-2 pandemic has shown, human populations are
highly heterogeneous and it's hard to predict how even a suburb, let alone a
city or country, will be affected by an outbreak.

Nevertheless models, in particular agent-based ones, can provide insight into
infectious disease dynamics \cite{Geffen2018,Hunter2019}, and particularly the
effectiveness of CTI versus isolation only of infected people versus no contact
tracing or isolation at all. If simulations of human populations can show us
that under a wide variety of assumptions, CTI is likely to be beneficial then it
is worth implementing CTI in response to dangerous infection outbreaks.

\section{Methodology}

We simulated thousands of short duration illnesses with the following
model world:

\begin{itemize}

\item A population of $10,000$ agents is initiated with ten randomly infected
  agents. The remaining agents are in a \emph{Susceptible} state.

\item Infected agents are initially in an \emph{Exposed} state. They then
  advance to an \emph{Infectious Asymptomatic} state. Then they advance either
  to an \emph{Infectious Symptomatic} or \emph{Recovered} state. An agent that
  has advanced to the \emph{Infectious Symptomatic} stage advances to the
  \emph{Recovered} state. \footnote{While our simulation program can handle
  hospitalisation, intensive care and death states, we decided that this was
  unnecessary complexity for our purposes here.}

\item Uninfected agents may become exposed if they are adjacent to one of $k$
  neighbouring agents in one of the two infectious states. Each uninfected agent
  has its own susceptibility to being infected by its neighbours. Each infected
  agents also has its own level of infectiousness.

\item Agents are tested for the infection with a specified probability per day
depending on the state they are in. The test result becomes known an average
of a specified number of days later.


\item Agents that test positive may be placed in isolation for a fixed
number of days in which case they are less likely to become infected. Each agent
has its own adherence level to isolation which is factored into its risk of
being infected. There is a continuum of isolation adherence that affects the
risk of infection.

\item If an agent is one of the $k$ adjacent neighbours of an agent whose test
result is known it may be traced with a specified probability. If it is
successfully traced it is also placed into isolation.

\item We assume we are interested in highly contagious diseases with at most a
  single digit infection mortality rate, such as SARS-CoV-2 or influenza, and
  unlike MERS or Ebola (though our work can easily be extended to examine these
  as well). Hence we have not factored death into the simulations.

\end{itemize}

Thus the risk of infection of an uninfected agent, $a$, that comes into contact
with an infectious one, $b$, is a stochastic function of the susceptibility to
infection of $a$, the infectiousness of $b$ and the isolation of $a$ and $b$.

A simulation runs as follows: First the agents are initialized. Then for 500
iterations, where each iteration represents a day, the following events take
place: new infections, testing, isolation, de-isolation (for agents who have
been isolated for the specified number of days), tracing, and finally disease
progression.

The simulation engine has the algorithmic form described in Listing
~\ref{listing:engine}.

\begin{minipage}{\linewidth}

\begin{lstlisting}[caption=Structure of the simulation engine, label=listing:engine]

  Initialize Agents
  For each day of the simulation:
     For each event
        Select the agents to whom the event applies
        Apply the event
     End
  End
\end{lstlisting}

\end{minipage}

This structure is based on \cite{Geffen2017}.

By design our model, named HETGEN, encompasses extremely heterogeneous and
stochastic agent behaviour: (1) each agent has its own susceptibility when
uninfected, and infectiousness when infected, (2) infected agents traverse
through the infection stages stochastically, (3) agents have their own isolation
adherence parameter, (4) agents get tested stochastically but infected
symptomatic agents are much more likely to get tested, (5) tracing occurs when
an infected agent's test is returned with a positive result, and (6) tracing,
too is stochastic, with a success probability for each contact.

Listing ~\ref{listing:infection} is the pseudo-code for the infection algorithm.

\begin{lstlisting}[mathescape=true, caption=Infection algorithm, label=listing:infection]
  For each infected agent, $a$
     For each agent $b$ where $b$ is one of up to $k/2$ neighbors on either side of $a$
        if $b$ is uninfected
           Generate a uniform random number between $0$ and $1$, $r$
           $x$ = $f$(infectiousness of $a$, isolation adherence of $a$ if isolated,
                    susceptibility of $b$, isolation adherence of $b$ if isolated)
           if $r < x$
              Set $b$'s state to Exposed
           End
        End
     End
  End
\end{lstlisting}


The infection function, $f$ in Listing ~\ref{listing:infection} for two agents,
$a$ and $b$, where $a$ is infectious and $b$ is susceptible, $i$ is a property
between $0$ and $1$ of $a$ and $b$ measuring their adherence to isolation, $t$
is the infectious of $a$, and $s$ is the susceptibility of $b$, is
as follows:

\begin{equation}
min(1-a_i,1-b_i) (a_t+b_s)/2
\end{equation}

This unstructured model is designed to contrast, and complement, a highly
structured model we have previously described that was designed to capture
specific characteristics of Covid-19 and the effect of CTI on reducing
SARS-CoV-2 infections in a South African township in which some agents attend
schools, some work, some use taxis, and all live in households which are located
in neighbourhood blocks, with each of these settings conferring different
risks.\cite{Low2020}

\subsection{Scenarios}

We compared five scenarios:

\begin{description}

\item[None] There is no isolation or contact tracing.

\item[Isolation] There is only isolation of agents with positive test results,
  with 85\% mean adherence per day.

\item[Minimum] There is isolation of agents with positive tests results with
  85\% mean adherence per day, and 10\% of their contacts are traced and
  isolated with 85\% mean adherence per day.

\item[Moderate] There is isolation of agents with positive results with 85\%
  adherence per day, and 30\% of their contacts are traced and isolated with
  85\% mean adherence per day.

\item[Maximum] There is perfect isolation of agents when they test positive and
  perfect tracing of contacts of their contacts, who are then also perfectly
  isolated.

\end{description}

We ran $15,000$ sensitivity tests, repeated $10$ times for each scenario, on
$10,000$ agents. This comes to $15,000 \times 10 \times 5 = 750,000$
simulations.

On each sensitivity test several of the parameters are randomly perturbed, using
a uniform distribution, over a specified range. Table ~\ref{table:parameters}
lists the parameters used in the model. Entries in the value column separated by
a hyphen are the perturbed parameters, while those that do no have a hyphen are
held constant across all simulations.

\begin{table}[h!]
\begin{tabular}{|p{7cm}|c|p{7cm}|}
  \hline
Parameter	&Value	&Notes\\ \hline
Number of days	& 500	&Each simulation iteration corresponds to 1 day\\ \hline
Initial infections	& 10	&Infected agents are set to Exposed\\ \hline
Daily probability of Susceptible agent getting tested	& 0	&\\ \hline
Daily probability of Exposed agent getting tested 	& 0.1-0.3	&\\ \hline
Daily probability of Infectious Asymptomatic agent getting tested	&0.1-0.3	&\\ \hline
Daily probability of Infectious Symptomatic agent getting tested	& 0.3-0.9	&\\ \hline
Probability of exposed agent test returning positive	& 0.5	&\\ \hline
Probability of Infectious Asymptomatic agent test returning positive	& 0.9	&\\ \hline
Probability of Infectious Symptomatic agent test returning positive	& 0.999	&\\ \hline
Mean number of days for test result to come back	& 1-6	&\\ \hline
Minimum number of days for test result to come back	& 0-2	&Even if mean is less than this, this prevails\\ \hline
Isolation days	& 6-14	&\\ \hline
Probability per day of staying Exposed	& 0.1-0.9	&\\ \hline
Probability per day of staying Infectious Asymptomatic	& 0.1-0.9	&\\ \hline
Probability of changing from Infectious Asymptomatic to Recovered	& 0.05-0.95	&\\ \hline
Probability per day of staying Infectious Symptomatic	& 0.1-0.9	&\\ \hline
Number of neighbours affected in matching and isolation	& 32-44	&\\
\hline
\end{tabular}
\caption{Parameters used in the simulations. Entries in the Value column with a
  hyphen are randomly perturbed before each sensitivity test.}
\label{table:parameters}
\end{table}

\subsection{Limitations}

While the model encompasses a wide variety of variables for a wide number of
possible illnesses, it is merely a computer simulation and cannot capture all
real-world dynamics. Also the search space of possible illnesses, even with the
limited number of variables we have perturbed, is massive (indistinguishable
from infinity for practical purposes). $15,000$ illnesses represents a tiny
fraction of the search space. Nevertheless our sample should be large enough to
draw tentative conclusions that inform policy, or at least further clinical
research.

A further limitation of the infection algorithm is that only the agents in the
neighbourhood of an infectious agent can become infected. This algorithm, while
more realistic than random or unassortative mixing ones, certainly doesn't
capture the complexity of real-world contacts. In future work we will explore
compromises between assortative and unassortative mixing.

Also there is no migration by agents into or out of our model-world.

While this work may suggest that CTI can mitigate the spread of infectious
disease, it offers no insight into how it can be implemented effectively.

\subsection{Programming}

Our model was prototyped in Python and then recoded, and further developed, in
C++. This allowed us to run hundreds of thousands of simulations in several
hours on affordable, standard consumer hardware. We recommend this methodology
to other modellers who are welcome to use our code as a basis for this kind of
simulation. The HETGEN source is available under the GNU General Public License
version 3.0. Our code and results are available at \url{https://github.com/nathangeffen/ABM_CTI}.

\subsection{Calibration, $R_0$ and wide range of infections}

To the extent that the model has been calibrated, it has been done
experimentally. The values in Table ~\ref{table:parameters} that are perturbed
in $15,000$ sensitivity tests, have been set to go slightly beyond the range of
reasonable estimates for influenza and SARS-CoV-2 infections. Those that are not
perturbed have been set to what we hope are reasonable values.

Because a practically infinite number of illnesses are encompassed in these
values, many outbreaks do not have an $R_0$ above 1 and fizzle out
immediately. Others, on the other hand, have extremely high $R_0$ values, with
every agent becoming infected if no intervention takes place. Calculating $R_0$
per simulation for such models in which the agents behave so heterogeneously is
also not useful, with different methodologies giving widely different
estimates. We therefore do an analysis of the results that includes all the
simulations and sub-analyses that exclude those where fewer than 500 agents
become infected.

\section{Results}\label{results}

We calculated the mean, median and standard deviation over the ten runs of each
of the $15,000$ sensitivity test for each of the five scenarios. Table
~\ref{table:results_10k} shows the results.

As expected, the \emph{None} scenario had the highest mean and median total
infections, followed by \emph{Isolation}, \emph{Minimum}, \emph{Moderate} and
\emph{Maximum}.

Note that the median for the \emph{None} scenario is higher than the mean,
indicating that the mean is pulled down disproportionately by illnesses with
low $R_0$ that never become sizeable. By contrast the medians of the three most
effective intervention scenarios are lower than the means, indicating that the
means are increased disproportionately by epidemics with large $R_0$ that,
despite the intervention, still resulted in a large number of infections.

\begin{table}[h!]
  \centering
  \begin{tabular}{|l|c|c|c|}
    \hline
    Scenario & Mean & Median & Std\\ \hline
    None & $5,743$ & $7,511$ & $4,269$\\ \hline
    Isolation & $5,341$ & $5,983$ & $4,285$\\ \hline
    Minimum & $4,419$ & $2,991$ & $4,092$ \\ \hline
    Moderate & $3,359$ & $1,282$ & $3,731$ \\ \hline
    Maximum & $994$ & $273$ & $1,716$\\ \hline
  \end{tabular}
  \caption{Results of the $15,000$ sensitivity tests using $10,000$ agents.}
  \label{table:results_10k}
\end{table}

Also useful is to know for how many illnesses each scenario outperformed the
others. We compared scenarios using the mean of the ten runs for each
illness. If an intervention has no effect, it should be outperformed by the None
scenario approximately $7,500$ times (i.e. half of the $15,000$
illnesses). Using the mean of the ten runs per illness, the None scenario only
outperformed the Isolation, Minimum, Moderate and Maximum interventions $2,986$,
$828$, $423$ and $193$ times respectively. (Note: Whenever the None scenario beats
another scenario, it is only due to the stochastic nature of the simulations.)

Many illnesses never reach epidemic proportions or the difference between the
interventions is small. We therefore did a further analysis of counting the
number of times each scenario outperformed (or was outperformed by) the no
intervention scenario if two criteria were met: (1) A mean of at least 500
agents were infected in the None scenario across the tens runs of an illness and
(2) the better performing scenario had no more than 70\% of the infections of
the scenario it was being compared to. Using these criteria the None scenario
only outperformed the Isolation, Minimum, Moderate and Maximum interventions $28$,
$2$, $0$ and $0$ times respectively. It was outperformed by them $2,020$, $5,406$,
$7,754$ and $11,392$ times respectively. Making the second criterion even more
stringent, by lowering it to 40\%, resulted in the None scenario never
outperforming any of the intervention scenarios. The intervention scenarios
outperformed the None scenario $346$, $2,599$, $5,657$ and $10,585$ times
respectively.

\section{Conclusions}\label{conclusions}

Our analysis suggests that if an infectious outbreak occurs with certain
characteristics similar to influenza and SARS-CoV-2, even moderately well
implemented CTI is likely to substantially reduce the ultimate
number of infections. These characteristics are: (1) There must be a test for
the illness with reasonable specificity and sensitivity for which results can be
obtained in a few days. (2) A significant number of infectious people should
present for the test, (3) the length of the illness should exceed the mean
turnaround time for the test, and (4) the means to carry out contact tracing and
isolation must exist. These characteristics are achievable for seasonal
influenza and SARS-CoV-2 in many countries.

Even moderately implemented isolation measures without tracing are likely to
have a substantial benefit. This suggests that if people are, for example,
strongly encouraged to stay at home if they have influenza symptoms during the
influenza season, many infections, and consequently a substantial number of
deaths, can be averted, even more so perhaps if entire households remain at home
while one member is symptomatic with influenza. It's conceivable that this may
even reduce the total number of days absent from work during influenza season,
though further research is needed to test this hypothesis.

Randomized cluster-controlled clinical trials conceivably could test the
efficacy of contact tracing. A Cochrane Review has pointed out the lack of
contact tracing randomised trials for tuberculosis \cite{Brangaza2019}. This gap
in medical evidence applies to most infectious diseases. While models such as
ours suggest that it is worthwhile investing in contact tracing infrastructure
for infectious diseases, randomised trials can offer much clearer insight into
the cost-effectiveness of CTI and its practical feasibility than models.

Interestingly, because our model is highly heterogeneous and our analysis finds
that, occasionally, even with the best possible implementation of CTI, there
will be no benefit merely due to stochastic effects. We suspect this is a
finding with real-world relevance. It is likely that in two similar settings,
wherein no intervention is implemented in one, while a comprehensive CTI
intervention is implemented in the other, the latter will, in a minority of
occasions, have a worse epidemic, solely due to the stochastic nature of
epidemics.

\begin{thebibliography}{1}

\bibitem{Flaxman2020}
  Flaxman S, Mishra S, Gandy A, et al.
  \newblock Estimating the effects of non-pharmaceutical interventions on
  COVID-19 in Europe.
  \newblock Nature. 2020;584(7820):257-261.
  \newblock doi:10.1038/s41586-020-2405-7

\bibitem{Lemaitre2020}
  Lemaitre JC, Perez-Saez J, Azman AS, Rinaldo A, and Fellay J.
  \newblock Assessing the impact of non-pharmaceutical interventions on
  SARS-CoV-2 transmission in Switzerland.
  \newblock Swiss Med Wkly. 2020;150:w20295. Published 2020 May 30.
  \newblock doi:10.4414/smw.2020.20295

\bibitem{Cowling2020}
  Cowling BJ, Ali ST, Ng TWY, et al.
  \newblock Impact assessment of non-pharmaceutical interventions against
  coronavirus disease 2019 and influenza in Hong Kong: an observational study.
  \newblock Lancet Public Health. 2020;5(5):e279-e288.
  \newblock doi:10.1016/S2468-2667(20)30090-6

\bibitem{Jung2020}
  Jung J, Hong MJ, Kim EO, Lee J, Kim MN, Kim SH.
  \newblock Investigation of a nosocomial outbreak of coronavirus disease 2019
  in a paediatric ward in South Korea: successful control by early detection and
  extensive contact tracing with testing [published online ahead of print, 2020
    Jun 25].
  \newblock Clin Microbiol Infect. 2020;S1198-743X(20)30365-7.
  \newblock doi:10.1016/j.cmi.2020.06.021

\bibitem{Lee2020}
  Lee SW, Yuh WT, Yang JM, et al.
  \newblock Nationwide Results of COVID-19 Contact Tracing in South Korea:
  Individual Participant Data From an Epidemiological Survey.
  \newblock JMIR Med Inform. 2020;8(8):e20992. Published 2020 Aug 25.
  \newblock doi:10.2196/20992

\bibitem {Park2016}
  Park GE, Ko JH, Peck KR, et al.
  \newblock Control of an Outbreak of Middle East Respiratory Syndrome in a
  Tertiary Hospital in Korea.
  \newblock Ann Intern Med. 2016;165(2):87-93.
  \newblock doi:10.7326/M15-2495

\bibitem{Dennis2018}
  Dennis AM, Pasquale DK, Billock R, et al.
  \newblock Integration of Contact Tracing and Phylogenetics in an Investigation
  of Acute HIV Infection.
  \newblock Sex Transm Dis. 2018;45(4):222-228.
  \newblock doi:10.1097/OLQ.0000000000000726

\bibitem{MacPherson2019}
  MacPherson P, Webb EL, Variava E, et al.
  \newblock Intensified household contact tracing, prevention and treatment
  support versus enhanced standard of care for
  contacts of tuberculosis cases in South Africa: study protocol for a household
  cluster-randomised trial.
  \newblock BMC Infect Dis. 2019;19(1):839. Published 2019 Oct 12.
  \newblock doi:10.1186/s12879-019-4502-5

\bibitem{Katzman2019}
  Katzman C, Mateu-Gelabert P, Kapadia SN, Eckhardt BJ.
  \newblock Contact tracing for hepatitis C: The case for novel screening
  strategies as we strive for viral elimination.
  \newblock Int J Drug Policy. 2019;72:33-39.
  \newblock doi:10.1016/j.drugpo.2019.04.003

\bibitem{Stokes1999}
  Stokes T, Schober P.
  \newblock A survey of contact tracing practice for sexually transmitted
  diseases in GUM clinics in England and Wales.
  \newblock Int J STD AIDS. 1999;10(1):17-21.
  \newblock doi:10.1258/0956462991913024

  \bibitem{Swaan2011}
    Swaan CM, Appels R, Kretzschmar ME, van Steenbergen JE.
    \newblock Timeliness of contact tracing among flight passengers for
    influenza A/H1N1 2009.
    \newblock BMC Infect Dis. 2011;11:355. Published 2011 Dec 28.
    \newblock doi:10.1186/1471-2334-11-355

\bibitem{Eames2010}
  Eames KT, Webb C, Thomas K, Smith J, Salmon R, Temple JM.
  \newblock Assessing the role of contact tracing in a suspected H7N2 influenza
  A outbreak in humans in Wales.
  \newblock BMC Infect Dis. 2010;10:141. Published 2010 May 28.
  \newblock doi:10.1186/1471-2334-10-141


\bibitem{Swanson2018}
  Swanson KC, Altare C, Wesseh CS, et al.
  \newblock Contact tracing performance during the Ebola epidemic in Liberia,
  2014-2015.
  \newblock PLoS Negl Trop Dis. 2018;12(9):e0006762. Published 2018 Sep 12.
  \newblock doi:10.1371/journal.pntd.0006762


\bibitem{Geffen2017}
  Geffen N, Scholz SM.
  \newblock Efficient and Effective Pair-Matching Algorithms for Agent-Based
  Models.
  \newblock Journal of Artificial Societies and Social Simulation 20 (4) 8,
  2017.
  \newblock doi: 10.18564/jasss.3485


\bibitem{Geffen2018}
  Geffen N, Scholz SM.
  \newblock How various design decisions on matching individuals in
  relationships affect the outcomes of microsimulations of sexually transmitted
  infection epidemics.
  \newblock PLoS One. 2018;13(8):e0202516. Published 2018 Aug 29.
  \newblock doi:10.1371/journal.pone.0202516

\bibitem{Hunter2019}
  Hunter E, Mac Namee B, Kelleher J.
  \newblock An open-data-driven agent-based model to simulate infectious disease
  outbreaks [published correction appears in PLoS One.
    \newblock 2019 Jan 17;14(1):e0211245]. PLoS
  One. 2018;13(12):e0208775. Published 2018 Dec 19.
  \newblock doi:10.1371/journal.pone.0208775

\bibitem{Brangaza2019}
  Braganza Menezes D, Menezes B, Dedicoat M.
  \newblock Contact tracing strategies in household and congregate environments
  to identify cases of tuberculosis in low- and moderate-incidence populations.
  \newblock Cochrane Database Syst Rev. 2019;8(8):CD013077. Published 2019 Aug 28.
  \newblock doi:10.1002/14651858.CD013077.pub2

\bibitem{Low2020}
  Low M, Geffen N.
  \newblock Contact tracing and isolation reduces Covid-19 incidence in a
  structured agent-based model.
  \newblock 6 October 2020. medRxiv 2020.10.06.20207761; doi: https://doi.org/10.1101/2020.10.06.20207761

\end{thebibliography}


\end{document}
